\documentclass[11pt]{article}
\usepackage[margin=1in]{geometry}
\usepackage{graphicx}
\usepackage{amsmath}
\usepackage{booktabs}
\usepackage{float}
\usepackage{hyperref}

\title{CV701 Assignment 4 -- Task 1 Report}
\author{Mashrafi Monon}
\date{\today}

\begin{document}
\maketitle

\section{Introduction}
This report summarizes the implementation and evaluation of the facial keypoint detection model for Task~1. The dataset consists of celebrity faces with 68 annotated landmarks. Our objective is to regress the landmark coordinates and infer the facial emotion (negative/neutral/positive) from the detected keypoints.

\section{Methodology}
\subsection{Data Pipeline}
Describe the dataset splits, custom transforms (resize, random flip, normalization), and dataloader configuration.

\subsection{Model}
Summarize the ResNet18-based regression head, training hyperparameters, loss (Smooth L1), optimizer (AdamW), and scheduler (Cosine Annealing).

\subsection{Emotion Heuristic}
Explain the rule-based classifier operating on predicted landmarks (mouth width/height, smile curvature normalized by inter-ocular distance).

\section{Experiments}
\subsection{Training Configuration}
List key hyperparameters (batch size, epochs, learning rate, device).

\subsection{Validation Metrics}
Present MAE/RMSE/NME curves, mention best epoch, and reference figures/tables.

\subsection{Test Results}
Summarize the metrics stored in \texttt{artifacts/task1\_hpc/metrics.json} and emotion distribution from the test predictions CSV.

\section{Qualitative Analysis}
Include visualizations or descriptions of correctly predicted landmarks and common failure cases (e.g., occlusions, profile faces).

\section{Conclusion and Future Work}
Wrap up Task~1 highlights and outline next steps for Task~2 (deployment + optimization).

\bibliographystyle{plain}
\bibliography{references}

\end{document}
